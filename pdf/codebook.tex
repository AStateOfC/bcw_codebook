\documentclass[10pt,twocolumn,oneside]{article}
\setlength{\columnsep}{10pt}                                                                    %兩欄模式的間距
\setlength{\columnseprule}{0pt}                                                                %兩欄模式間格線粗細

\usepackage{amsthm}								%定義,例題
\usepackage{amssymb}
%\usepackage[margin=2cm]{geometry}
\usepackage{fontspec}								%設定字體
\usepackage{color}
\usepackage[x11names]{xcolor}
\usepackage{xeCJK}								%xeCJK
\usepackage{listings}								%顯示code用的
%\usepackage[Glenn]{fncychap}						%排版,頁面模板
\usepackage{fancyhdr}								%設定頁首頁尾
\usepackage{graphicx}								%Graphic
\usepackage{enumerate}
\usepackage{titlesec}
\usepackage{amsmath}

%\usepackage[T1]{fontenc}
\usepackage{amsmath, courier, listings, fancyhdr, graphicx}
\topmargin=0pt
\headsep=5pt
\textheight=780pt
\footskip=0pt
\voffset=-40pt
\textwidth=545pt
\marginparsep=0pt
\marginparwidth=0pt
\marginparpush=0pt
\oddsidemargin=0pt
\evensidemargin=0pt
\hoffset=-42pt

%\renewcommand\listfigurename{圖目錄}
%\renewcommand\listtablename{表目錄} 

%%%%%%%%%%%%%%%%%%%%%%%%%%%%%

\setmainfont{Consolas}				%主要字型
\setCJKmainfont{Consolas}			%中文字型
\XeTeXlinebreaklocale "zh"						%中文自動換行
\XeTeXlinebreakskip = 0pt plus 1pt				%設定段落之間的距離
\setcounter{secnumdepth}{3}						%目錄顯示第三層

%%%%%%%%%%%%%%%%%%%%%%%%%%%%%
\makeatletter
\lst@CCPutMacro\lst@ProcessOther {"2D}{\lst@ttfamily{-{}}{-{}}}
\@empty\z@\@empty
\makeatother
\lstset{											% Code顯示
language=C++,										% the language of the code
basicstyle=\footnotesize, 						% the size of the fonts that are used for the code
%numbers=left,										% where to put the line-numbers
numberstyle=\footnotesize,						% the size of the fonts that are used for the line-numbers
stepnumber=1,										% the step between two line-numbers. If it's 1, each line  will be numbered
numbersep=5pt,										% how far the line-numbers are from the code
backgroundcolor=\color{white},					% choose the background color. You must add \usepackage{color}
showspaces=false,									% show spaces adding particular underscores
showstringspaces=false,							% underline spaces within strings
showtabs=false,									% show tabs within strings adding particular underscores
frame=false,											% adds a frame around the code
tabsize=2,											% sets default tabsize to 2 spaces
captionpos=b,										% sets the caption-position to bottom
breaklines=true,									% sets automatic line breaking
breakatwhitespace=false,							% sets if automatic breaks should only happen at whitespace
escapeinside={\%*}{*)},							% if you want to add a comment within your code
morekeywords={*},									% if you want to add more keywords to the set
keywordstyle=\bfseries\color{Blue1},
commentstyle=\itshape\color{Red4},
stringstyle=\itshape\color{Green4},
}

%%%%%%%%%%%%%%%%%%%%%%%%%%%%%

\begin{document}
\pagestyle{fancy}
\fancyfoot{}
%\fancyfoot[R]{\includegraphics[width=20pt]{ironwood.jpg}}
\fancyhead[L]{National Taiwan University bcw0x1bd2}
\fancyhead[R]{\thepage}
\renewcommand{\headrulewidth}{0.4pt}
\renewcommand{\contentsname}{Contents} 

\scriptsize
\tableofcontents
%%%%%%%%%%%%%%%%%%%%%%%%%%%%%
\section{Basic}
\subsection{.vimrc}
\lstinputlisting{../codes/Basic/.vimrc}

\newpage

\subsection{IncreaseStackSize}
\begin{lstlisting}
//stack resize
asm( "mov %0,%%esp\n" ::"g"(mem+10000000) );
//change esp to rsp if 64-bit system

//stack resize (linux)
#include <sys/resource.h>
void increase_stack_size() {
	const rlim_t ks = 64*1024*1024;
	struct rlimit rl;
	int res=getrlimit(RLIMIT_STACK, &rl);
	if(res==0){
		if(rl.rlim_cur<ks){
			rl.rlim_cur=ks;
			res=setrlimit(RLIMIT_STACK, &rl);
		}
	}
}
\end{lstlisting}

\subsection{Default Code}
\lstinputlisting{../codes/Basic/default.cpp}
\newpage

\section{Data Structure}
\subsection{Bigint}
\lstinputlisting{../codes/Data_Structure/Bigint/Bigint.cpp}

\subsection{Leftist Heap}
\lstinputlisting{../codes/Data_Structure/Leftist_Heap/Leftist_Heap.cpp}
\newpage

\subsection{extc\_balance\_tree}
\lstinputlisting{../codes/Data_Structure/Balance_Tree/extc_bt.cpp}

\subsection{Treap}
\lstinputlisting{../codes/Data_Structure/Balance_Tree/treap.cpp}

\section{Graph}

\subsection{Tarjan}
\lstinputlisting{../codes/Graph/Tarjan/Tarjan.cpp}

\subsection{Strongly Connected Components:Kosaraju's Algorithm}
\lstinputlisting{../codes/Graph/scc/kosaraju.cpp}

%\subsection{DMST}
%\lstinputlisting{../codes/Graph/dmst/dmst.cpp}

\subsection{DMST\_with\_sol}
\lstinputlisting{../codes/Graph/dmst/dmst_sol.cpp}

\section{Flow}
\subsection{ISAP} %Checked
\lstinputlisting{../codes/Graph/Flow/isap.cpp}

\subsection{Bipartite Matching (Augmenting Path)}
\lstinputlisting{../codes/Graph/Matching/augmenting_path.cpp}

\subsection{SW-Mincut}
\lstinputlisting{../codes/Graph/Flow/SW-mincut.cpp}

%TODO Wrong
\subsection{Maximum Simple Graph Matching}
\lstinputlisting{../codes/Graph/Matching/Maximum_Simple_Graph_Matching.cpp}

\subsection{2-Commodity Flow}
\lstinputlisting{../codes/Graph/Flow/2comFlow.cpp}

\section{Math}
\subsection{ax+by=gcd}
\lstinputlisting{../codes/Math/ax+by=gcd/ax+by=gcd.cpp}

\subsection{Chinese Remainder}
\lstinputlisting{../codes/Math/Chinese_Remainder/Chinese_Remainder.cpp}

\subsection{Miller Rabin}
\lstinputlisting{../codes/Math/Miller_Rabin/Miller_Rabin.cpp}

\subsection{Mod}
\lstinputlisting{../codes/Math/MOD/MOD.cpp}

\subsection{Primes}
\lstinputlisting{../codes/Math/primes.cpp}

\section{Geometry}
\subsection{Point operators}
\lstinputlisting{../codes/Geometry/Point_operators/Point_operators.cpp}

\subsection{Minimum Covering Circle}
\lstinputlisting{../codes/Geometry/Minimum_covering_circle/mcc.cpp}

\subsection{Intersection of two circles}
Let $\mathbf{O_1} = (x_1, y_1), \mathbf{O_2} = (x_2, y_2)$ be two centers of circles,  $r_1,r_2$ be the radius. If:\\
$ d = | \mathbf{O_1} - \mathbf{O_2} | $
$ \mathbf{u} = \frac{ 1 }{ 2 } ( \mathbf{O_1} + \mathbf{O_2} )+ 
\frac{ ( r_2^2 - r_1^2 )  }
{ 2 d^2 } ( \mathbf{O_1} - \mathbf{O_2} )$ \\
$ \mathbf{v} = \frac{ \sqrt{(r_1+r_2+d) (r_1-r_2+d) (r_1+r_2-d) (-r_1+r_2+d)} }{ 2 d^2 } ( y_1 - y_2 , -x_1 + x_2 )$
then $ \mathbf{u} + \mathbf{v} , \mathbf{u} - \mathbf{v} $
are the two intersections of the circles, provided that $ d < r_1 + r_2 $.



\subsection{Intersection of two lines}
\lstinputlisting{../codes/Geometry/Intersection_of_two_lines/Intersection_of_two_lines.cpp}

%TODO wrong when intersection is empty
\subsection{Half line Intersection}
\lstinputlisting{../codes/Geometry/Half_line_intersection/Half_line_intersection.cpp}

\section{Stringology}
\subsection{Suffix Array}
\lstinputlisting{../codes/Stringology/Suffix_Array/Suffix_Array.cpp}

\subsection{Suffix Array (SAIS TWT514)}
\lstinputlisting{../codes/Stringology/Suffix_Array/sais_twt514.cpp}

\subsection{Aho-Corasick Algorithm}
\lstinputlisting{../codes/Stringology/Automata/Aho-Corasick.cpp}

\subsection{Z value}
\lstinputlisting{../codes/Stringology/Z_Value/zvalue.cpp}

\subsection{Z value (palindrome ver.)}
\lstinputlisting{../codes/Stringology/Z_Value/zvalue_palindrome.cpp}

%TODO update to vector version
\subsection{Suffix Automaton}
\lstinputlisting{../codes/Stringology/Automata/SAM.cpp}

\section{Problems}
\subsection {Qtree IV}
\lstinputlisting{../codes/Problem/QTREE4.cpp}

\subsection{Find the maximun tangent (x,y is increasing)}
\lstinputlisting{../codes/Problem/maxtan.cpp}


\section{+1ironwood's code}
\subsection{KDTreeAndNearestPoint}
\begin{lstlisting}
#define INF 1100000000
class NODE{ public:
	int x,y,x1,x2,y1,y2;
	int i,f;
	NODE *L,*R;
};
inline long long dis(NODE& a,NODE& b){
	long long dx=a.x-b.x;
	long long dy=a.y-b.y;
	return dx*dx+dy*dy;
}
NODE node[100000];
bool cmpx(const NODE& a,const NODE& b){ return a.x<b.x; }
bool cmpy(const NODE& a,const NODE& b){ return a.y<b.y; }
NODE* KDTree(int L,int R,int dep){
	if(L>R) return 0;
	int M=(L+R)/2;
	if(dep%2==0){
		nth_element(node+L,node+M,node+R+1,cmpx);
		node[M].f=0;
	}else{
		nth_element(node+L,node+M,node+R+1,cmpy);
		node[M].f=1;
	}
	node[M].x1=node[M].x2=node[M].x;
	node[M].y1=node[M].y2=node[M].y;
	node[M].L=KDTree(L,M-1,dep+1);
	if(node[M].L){
		node[M].x1=min(node[M].x1,node[M].L->x1);
		node[M].x2=max(node[M].x2,node[M].L->x2);
		node[M].y1=min(node[M].y1,node[M].L->y1);
		node[M].y2=max(node[M].y2,node[M].L->y2);
	}
	node[M].R=KDTree(M+1,R,dep+1);
	if(node[M].R){
		node[M].x1=min(node[M].x1,node[M].R->x1);
		node[M].x2=max(node[M].x2,node[M].R->x2);
		node[M].y1=min(node[M].y1,node[M].R->y1);
		node[M].y2=max(node[M].y2,node[M].R->y2);
	}
	return node+M;
}
inline int touch(NODE* r,int x,int y,long long d){
	long long d2;
	d2 = (long long)(sqrt(d)+1);
	if(x<r->x1-d2 || x>r->x2+d2 || y<r->y1-d2 || y>r->y2+d2)
		return 0;
	return 1;
}
void nearest(NODE* r,int z,long long &md){
	if(!r || !touch(r,node[z].x,node[z].y,md)) return;
	long long d;
	if(node[z].i!=r->i){
		d=dis(*r,node[z]);
		if(d<md) md=d;
	}
	if(r->f==0){
		if(node[z].x<r->x){
			nearest(r->L,z,md);
			nearest(r->R,z,md);
		}else{
			nearest(r->R,z,md);
			nearest(r->L,z,md);
		}
	}else{
		if(node[z].y<r->y){
			nearest(r->L,z,md);
			nearest(r->R,z,md);
		}else{
			nearest(r->R,z,md);
			nearest(r->L,z,md);
		}
	}
}
int main(){
	int TT,n,i;
	long long d;
	NODE* root;
	scanf("%d",&TT);
	while(TT--){
		scanf("%d",&n);
		for(i=0;i<n;i++){
			scanf("%d %d",&node[i].x,&node[i].y);
			node[i].i=i;
		}
		root=KDTree(0,n-1,0);
		for(i=0;i<n;i++){
			d=9000000000000000000LL;
			nearest(root,i,d);
			ans[node[i].i]=d;
		}
	}
}
\end{lstlisting}
\subsection{MinkowskiSum}
\begin{lstlisting}
/* convex hull Minkowski Sum*/
#define INF 100000000000000LL
class PT{ public:
	long long x,y;
	int POS(){
		if(y==0) return x>0?0:1;
		return y>0?0:1;
	}
};
PT pt[300000],qt[300000],rt[300000];
long long Lx,Rx;
int dn,un;
inline bool cmp(PT a,PT b){
	int pa=a.POS(),pb=b.POS();
	if(pa==pb) return (a^b)>0;
	return pa<pb;
}
int minkowskiSum(int n,int m){
	int i,j,r,p,q,fi,fj;
	for(i=1,p=0;i<n;i++){
		if(pt[i].y<pt[p].y || (pt[i].y==pt[p].y && pt[i].x<
					pt[p].x)) p=i; }
	for(i=1,q=0;i<m;i++){
		if(qt[i].y<qt[q].y || (qt[i].y==qt[q].y && qt[i].x<
					qt[q].x)) q=i; }
	rt[0]=pt[p]+qt[q];
	r=1; i=p; j=q; fi=fj=0;
	while(1){
		if((fj&&j==q) || ((!fi||i!=p) && cmp(pt[(p+1)%n]-pt[
						p],qt[(q+1)%m]-qt[q]))){
			rt[r]=rt[r-1]+pt[(p+1)%n]-pt[p];
			p=(p+1)%n;
			fi=1;
		}else{
			rt[r]=rt[r-1]+qt[(q+1)%m]-qt[q];
			q=(q+1)%m;
			fj=1;
		}
		if(r<=1 || ((rt[r]-rt[r-1])^(rt[r-1]-rt[r-2]))!=0) r
			++;
		else rt[r-1]=rt[r];
		if(i==p && j==q) break;
	}
	return r-1;
}
void initInConvex(int n){
	int i,p,q;
	long long Ly,Ry;
	Lx=INF; Rx=-INF;
	for(i=0;i<n;i++){
		if(pt[i].x<Lx) Lx=pt[i].x;
		if(pt[i].x>Rx) Rx=pt[i].x;
	}
	Ly=Ry=INF;
	for(i=0;i<n;i++){
		if(pt[i].x==Lx && pt[i].y<Ly){ Ly=pt[i].y; p=i; }
		if(pt[i].x==Rx && pt[i].y<Ry){ Ry=pt[i].y; q=i; }
	}
	for(dn=0,i=p;i!=q;i=(i+1)%n){ qt[dn++]=pt[i]; }
	qt[dn]=pt[q]; Ly=Ry=-INF;
	for(i=0;i<n;i++){
		if(pt[i].x==Lx && pt[i].y>Ly){ Ly=pt[i].y; p=i; }
		if(pt[i].x==Rx && pt[i].y>Ry){ Ry=pt[i].y; q=i; }
	}
	for(un=0,i=p;i!=q;i=(i+n-1)%n){ rt[un++]=pt[i]; }
	rt[un]=pt[q];
}
inline int inConvex(PT p){
	int L,R,M;
	if(p.x<Lx || p.x>Rx) return 0;
	L=0;R=dn;
	while(L<R-1){ M=(L+R)/2;
		if(p.x<qt[M].x) R=M; else L=M; }
		if(tri(qt[L],qt[R],p)<0) return 0;
		L=0;R=un;
		while(L<R-1){ M=(L+R)/2;
			if(p.x<rt[M].x) R=M; else L=M; }
			if(tri(rt[L],rt[R],p)>0) return 0;
			return 1;
}
int main(){
	int n,m,i;
	PT p;
	scanf("%d",&n);
	for(i=0;i<n;i++) scanf("%I64d %I64d",&pt[i].x,&pt[i].y);
	scanf("%d",&m);
	for(i=0;i<m;i++) scanf("%I64d %I64d",&qt[i].x,&qt[i].y);
	n=minkowskiSum(n,m);
	for(i=0;i<n;i++) pt[i]=rt[i];
	scanf("%d",&m);
	for(i=0;i<m;i++) scanf("%I64d %I64d",&qt[i].x,&qt[i].y);
	n=minkowskiSum(n,m);
	for(i=0;i<n;i++) pt[i]=rt[i];
	initInConvex(n);
	scanf("%d",&m);
	for(i=0;i<m;i++){
		scanf("%I64d %I64d",&p.x,&p.y);
		p.x*=3; p.y*=3;
		puts(inConvex(p)?"YES":"NO");
	}
}
\end{lstlisting}
\subsection{MinimumMeanCycle}
\begin{lstlisting}
/* minimum mean cycle */
class Edge { public:
	int v,u;
	double c;
};
int n,m;
Edge e[MAXEDGE];
double d[MAXNUM][MAXNUM];
inline void relax(double &x,double val) { if(val<x) x=val; }
inline void bellman_ford() {
	int i,j;
	for(j=0;j<n;j++) d[0][j]=0.0;
	for(i=0;i<n;i++) {
		for(j=0;j<n;j++) d[i+1][j]=inf;
		for(j=0;j<m;j++)
			if(d[i][e[j].v]<inf-eps) relax(d[i+1][e[j].u],d[i][
					e[j].v]+e[j].c);
	}
}
inline double karp_mmc() {
	// returns inf if no cycle, mmc otherwise
	int i,k; double mmc=inf,avg;
	bellman_ford();
	for(i=0;i<n;i++) {
		avg=0.0;
		for(k=0;k<n;k++) {
			if(d[n][i]<inf-eps) avg=max(avg,(d[n][i]-d[k][i])/(
						n-k));
			else avg=max(avg,inf);
		}
		mmc=min(mmc,avg);
	}
	return mmc;
}
\end{lstlisting}

\subsection{PolynomialGenerator}
\begin{lstlisting}
class PolynomialGenerator {
	/* for a nth-order polynomial f(x), *
	 * given f(0), f(1), ..., f(n) *
	 * express f(x) as sigma_i{c_i*C(x,i)} */
	public:
		int n;
		vector<long long> coef;
		// initialize and calculate f(x), vector _fx should be
		filled with f(0) to f(n)
			PolynomialGenerator(int _n,vector<long long> _fx):n(_n
					),coef(_fx) {
				for(int i=0;i<n;i++)
					for(int j=n;j>i;j--)
						coef[j]-=coef[j-1];
			}
		// evaluate f(x), runs in O(n)
		long long eval(int x) {
			long long m=1,ret=0;
			for(int i=0;i<=n;i++) {
				ret+=coef[i]*m;
				m=m*(x-i)/(i+1);
			}
			return ret;
		}
};
\end{lstlisting}

\subsection{SwGeneralGraphMaxMatching}
\begin{lstlisting}
#define N 256 // max vertex num
class Graph { public:
	// n,g[i][j]=0/1, match() => match: (i,mate[i]) (or mate[i]=-1)
	int n, mate[N];
	bool g[N][N], inQ[N], inBlo[N];
	queue<int> Q;
	int start, newBase, prev[N], base[N];
	int lca(int u, int v) {
		bool path[N] = { false };
		while(true) {
			u = base[u]; path[u] = true;
			if(u == start) break;
			u = prev[mate[u]];
		}
		while(true) {
			v = base[v];
			if(path[v]) break;
			v = prev[mate[v]];
		}
		return v;
	}
	void trace(int u) {
		while(base[u] != newBase) {
			int v = mate[u];
			inBlo[base[u]] = inBlo[base[v]] = true;
			u = prev[v];
			if(base[u] != newBase) prev[u] = v;
		}
	}
	void contract(int u, int v) {
		newBase = lca(u, v);
		memset(inBlo, false, sizeof(inBlo));
		trace(u); trace(v);
		if(base[u] != newBase) prev[u] = v;
		if(base[v] != newBase) prev[v] = u;
		for(int i = 0; i < n; i++)
			if(inBlo[base[i]]) {
				base[i] = newBase;
				if(!inQ[i]) { Q.push(i); inQ[i] = true; }
			}
	}
	bool search() {
		memset(inQ, false, sizeof(inQ));
		memset(prev, -1, sizeof(prev));
		for(int i = 0; i < n; i++) base[i] = i;
		while(!Q.empty()) Q.pop();
		Q.push(start); inQ[start] = true;
		while(!Q.empty()) {
			int u = Q.front(); Q.pop();
			for(int i = 0; i < n; i++)
				if(g[u][i] && base[u] != base[i] && mate[u] != i){
					if(i == start || (mate[i] >= 0 && prev[mate[i]] >= 0)) contract(u, i);
					else if(prev[i] < 0) {
						prev[i] = u;
						if(mate[i] != -1) { Q.push(mate[i]); inQ[mate[i]] = true; }
						else { augment(i); return true; }
					}
				}
		}
		return false;
	}
	void augment(int u) {
		while(u >= 0) {
			int v = prev[u], w = mate[v];
			mate[v] = u; mate[u] = v; u = w;
		}
	}
	int match() {
		memset(mate, -1, sizeof(mate));
		int mth = 0;
		for(int i = 0; i < n; i++) {
			if(mate[i] >= 0) continue;
			start = i;
			if(search()) mth++;
		}
		return mth;
	}
};
\end{lstlisting}
\subsection{stoer-wagner-nm}
\begin{lstlisting}
// {{{ StoerWagner
const int inf=1000000000;
// should be larger than max.possible mincut
class StoerWagner {
	public:
		int n,mc; // node id in [0,n-1]
		vector<int> adj[MAXN];
		int cost[MAXN][MAXN];
		int cs[MAXN];
		bool merged[MAXN],sel[MAXN];
		// --8<-- include only if cut is explicitly needed
			DisjointSet djs;
		vector<int> cut;
		//--8<--------------------------------------------
			StoerWagner(int _n):n(_n),mc(inf),djs(_n) {
				for(int i=0;i<n;i++)
					merged[i]=0;
				for(int i=0;i<n;i++)
					for(int j=0;j<n;j++)
						cost[i][j]=cost[j][i]=0;
			}
		void append(int v,int u,int c) {
			if(v==u) return;
			if(!cost[v][u]&&c) {
				adj[v].PB(u);
				adj[u].PB(v);
			}
			cost[v][u]+=c;
			cost[u][v]+=c;
		}
		void merge(int v,int u) {
			merged[u]=1;
			for(int i=0;i<n;i++)
				append(v,i,cost[u][i]);
			// --8<-- include only if cut is explicitly needed
				djs.merge(v,u);
			//--8<--------------------------------------------
		}
		void phase() {
			priority_queue<pii> pq;
			for(int v=0;v<n;v++) {
				if(merged[v]) continue;
				cs[v]=0;
				sel[v]=0;
				pq.push(MP(0,v));
			}
			int v,s,pv;
			while(pq.size()) {
				if(cs[pq.top().S]>pq.top().F) {
					pq.pop();
					continue;
				}
				pv=v;
				v=pq.top().S;
				s=pq.top().F;
				pq.pop();
				sel[v]=1;
				for(int i=0;i<adj[v].size();i++) {
					int u=adj[v][i];
					if(merged[u]||sel[u]) continue;
					cs[u]+=cost[v][u];
					pq.push(MP(cs[u],u));
				}
			}
			if(s<mc) {
				mc=s;
				// --8<-- include only if cut is explicitly
				needed ------
					cut.clear();
				for(int i=0;i<n;i++)
					if(djs.getrep(i)==djs.getrep(v)) cut.PB(i);
				//--8<----------------------------------------
			}
			merge(v,pv);
		}
		int mincut() {
			if(mc==inf) {
				for(int t=0;t<n-1;t++)
					phase();
			
			return mc;
		}
		// --8<-- include only if cut is explicitly needed
		------
			vector<int> getcut() { // return one side of the cut
				mincut();
				return cut;
			}
		//--8<--------------------------------------------
};
// }}}
\end{lstlisting}

\end{document}
